\section{COMPASS Data Structures}
\label{section:data_structures}
To enable spatial search without relying on costly graph representations, \emph{COMPASS} encodes spatial entities using \textit{Location-Object} data structures and \textit{Object-Object} \textit{Concept Map} data structures. 
Our data structures are implemented hierarchically, with \textit{Objects} as the atomic elements. 
Objects are semantically grouped into \textit{Locations}, which are grouped into \textit{Regions}, which segment the world. 
For example, the `Washington DC' region might have `park' and `hospital' locations, and the `park' might have `swing', `slide', and `bench' objects, with directional relations defined between them. 
For a more detailed analysis of the motivation and methods for assigning objects to locations, see \textit{Geospatially Enhanced Search with Terrain Augmented Location Targeting} (\textbf{\textit{GESTALT}})~\cite{Osul2023}.

\subsection{Location-Object Data Structure}
A \textit{Location-Object} data structure is a per-location set-based representation of the directional relations between objects and their locations.
It subdivides the space around a location into the four cardinal quadrants (NorthWest, NorthEast, SouthWest, SouthEast) centered on the location coordinate.

Each location-object relationship is encoded independently of other objects at that location.
Figure \ref{figure:ConceptMap-LO} shows how we encode objects (\ref{fig:CM-LO-Example}) based on their relative position to the location coordinates (\ref{fig:CM-LO-Setup}).


\subsection{Object-Object Concept Map Data Structure}

An \textit{Object-Object concept map} (CM) is an $(n\times n)$ matrix representation that implicitly encodes the $m$ directional relations between $n$ stored objects, with similar intuition to image concept map representations~\cite{Xu2010}.
Matrix entries are either $0$, representing empty space, or an object ID, representing the relative position of that object to the $(n-1)$ other objects.
Using Algorithm~\ref{alg:geoToGrid}, we assign each object to a position $(i,j)$ in the matrix where $i$ is its order of appearance from north to south and $j$ is its order of appearance from west to east.
This method preserves the relative ordering of objects with respect to the original Cartesian representation.
Figure \ref{figure:ConceptMap} shows how a set of objects (\ref{fig:CM-Example}) are encoded in a concept map (\ref{fig:CM-OO-Setup}), preserving their directional relations. 


\begin{figure*}[t]
    \centering
    \begin{subfigure}[t]{.25\textwidth}
        \includegraphics[width=\textwidth]{CM-ExampleLocation.png}
        \caption{\small A candidate location X has named objects A-D with the spatial layout depicted above.} 
        \label{fig:CM-LO-Example}
    \end{subfigure}
    \hfill
    \begin{subfigure}[t]{.25\textwidth}
        \includegraphics[width=\textwidth]{CM-LO-Setup.png}
        \caption{\small The objects are binned into spatial quadrants based on their relative position to the location coordinates, X.} 
        \label{fig:CM-LO-Setup}
    \end{subfigure}
    \hfill
        \begin{subfigure}[t]{.25\textwidth}
        \includegraphics[width=\textwidth]{CM-LO-Query1.png}
        \caption{\small Rank the locations by the number of query terms found in the correct quadrant for the location.}
        \label{fig:CM-LO-Query}
    \hfill
    \end{subfigure}
    \caption{\textbf{Location-Object Search Method. A Location-Object data structure (Figure \ref{fig:CM-LO-Setup}) is generated based on the cardinal relations between the objects and the location (Figure \ref{fig:CM-LO-Example}). Then a pictorial query is matched against the structure (Figure \ref{fig:CM-LO-Query}).}}\label{figure:ConceptMap-LO} 
\end{figure*}





\begin{figure*}[h]
    \centering
    \begin{subfigure}[t]{.25\textwidth}
        \includegraphics[width=\textwidth]{CM-ExampleLocation.png}
        \caption{\small A candidate location X has named objects A-D with the spatial layout depicted above.}
        \label{fig:CM-Example}
    \end{subfigure}
    \hfill
    \begin{subfigure}[t]{.25\textwidth}
        \includegraphics[width=\textwidth]{CM-OO-Setup.png}
        \caption{\small Objects associated with location X are ordered from North to South (NS) and West to East (WE) and mapped into a matrix with corresponding indices.}
        \label{fig:CM-OO-Setup}
    \end{subfigure}
    \hfill
        \begin{subfigure}[t]{.25\textwidth}
        \includegraphics[width=\textwidth]{CM-OO-Query1.png}
        \caption{\small Recursive search. Each recursion is a darker shade, with an unpruned area in white. Objects highlighted in yellow are found to match the query configuration; candidate location X is a match for the query.}
        \label{fig:CM-OO-Query}
    \hfill
    \end{subfigure}
    \caption{\textbf{Object-Object Search Method. A Concept Map data structure (Figure \ref{fig:CM-OO-Setup}) is generated by ordering the objects associated with a candidate location (Figure \ref{fig:CM-Example}) from North to South and West to East. The search step (Figure \ref{fig:CM-OO-Query}) then recursively prunes the concept map until ANY matching configuration of objects is identified or the query constraints are not satisfied.
    }}\label{figure:ConceptMap} 
\end{figure*}

%This data structure subdivides the space around a location into segments or buckets, retaining a discrete set of directional relations, determined by the number of buckets.
%We choose to divide the space into four quadrants based on global cardinal direction with respect to the location- NorthWest, NorthEast, SouthWest, and SouthEast.