\section{Introduction}
\label{section:introduction}
\par{Many forms of spatial search require precise query input, like the coordinates or exact name of a location of interest, to return an accurate result. 
This formulation of the spatial search problem is cumbersome to users, as it requires queries to be issued in a format not inherently consistent with how humans reason spatially.
From a cognitive perspective, humans can more readily recall relative spatial layouts in their environment than they can estimate distances between objects or exact coordinates or addresses of places~\cite{Schwering2014, Weisberg2016, Miller2013, Keatley2021}.
As such, current methods in range querying and incremental nearest neighbor search require a user to specify distance thresholds to search for sets of objects, which is not ideal for a user with only partial information about their target location.
If the estimate is too large (small), the search will return too many (few) results. 
Further, these methods suffer in cases where there is a high overlap in terms, giving little discriminative power to searching for their presence alone (e.g., picnic tables, benches, and bus stops are common features of many locations). 
}
%
\par{Instead of forcing the human user to think like a map, a more natural approach is to have the 'map' organize itself to reflect how humans think. 
Some work has been done enabling users to specify spatial search queries in a pictorial format through a query interface~\cite{Soffer1997,DiLoreto1996} or mimicking how a human might draw a rough sketch map of a location from memory~\cite{Schwering2014}.
Although this approach allows a user to be much more expressive without needing to articulate their query using formal languages, pictorial querying invites significant computational challenges that have limited its use. 
Most pictorial methods formulate the search problem as a qualitative spatial reasoning constraint satisfaction problem (CSP), a set-based search problem, or a graph-matching problem, which scales prohibitively poorly for any realistic number of query constraint terms.
}

\par{We present \emph{COMPASS}, the \textit{Cardinal Orientation Manipulation and Pattern-Aware Spatial Search}, a collection of data structures and algorithms designed to address the scalability issues of pictorial querying by reframing the search problem, thus enabling a tractable, human-centric form of spatial search. 
\emph{COMPASS} extracts and explicitly encodes the implicit geographic relationships between objects and locations. 
By encoding these spatial relationships in a manner that can be compared with visual queries issued by the user, we enable a form of spatial search that (to our knowledge) does not presently exist in any GIS tools.
Specifically, we provide the following contributions:
\begin{enumerate}
    \item A reframing of the pictorial querying problem to include early stopping at the first set of objects that indicate a location matched the spatial configuration of objects queried
    \item A method for mapping objects associated with each location to one of two proposed spatial data structures that facilitate pictorially querying spatial relations like "A bus stop, pond, and telephone pole configured in a triangle."
    \item A series of algorithms to perform three forms of spatial search based on pictorial query input
    \item A substantial data generation framework that produces a location and object database and pictorial queries against it to enable comparisons of spatial search methods
\end{enumerate}

The paper proceeds in Section \ref{section:related} with a discussion of current and historical approaches to pictorial-based spatial search and spatial point data matching. 
Next, Section \ref{section:prelim} introduces our approach and the three forms of spatial search \emph{COMPASS} enables.
We describe the process used to generate the data, as well as the queries and algorithms that contributed to performing the search in Section \ref{section:methods}. 
Finally, in Section \ref{section:results}, we discuss the performance and scalability of the proposed methods.
}

%organizing geospatial data around the locations people can experience, cataloging the objects at those locations that they anchor their memories on, and abstracting away the need for the user to remember distances to objects, or their absolute cardinal positions we create an approach that focuses on the need of the human, not the convenient structure of the map.

%is a worthwhile endeavor for many Geographic Information System (GIS) applications.

%
%\par{The requirement to estimate distance is problematic in querying for locations, because the average person is terrible at estimating distances between things~\cite{Schwering2014}. 

%Consider a user trying to find a winery that they recall visiting on a tour recently.
%Wineries tend to be densely clustered in wine regions, and so unless the user knows the exact name of the winery, it can be difficult to find it with a keyword search because all the results will be geographically proximal.
%Users may remember some incomplete information about the winery they are trying to locate. 
%They might remember some of the \textit{objects} that are present at the winery (\textit{location}) they are looking for; a picnic table, a duck pond and a statue might come to mind.

%Even if humans weren't bad at judging distance, the heterogeneous nature of locations means that it is difficult to estimate a range or nearest neighbor query size that would work for the full range of locations in a given region. 
%For example, in many wine regions there will be a few big-name wineries and function centers that cover multiple acres, coexisting with small cellar doors operating out of someone's front garden, and the full spectrum between. 
%Our proposed solution is to take a human-centric approach to geospatial search.

%Instead of forcing the human to think like a map, we seek to have the 'map' organize itself to reflect how humans think. 
%While we are bad at judging distance, we are typically very good at remembering the relative configuration of objects in the space around us~\cite{Schwering2014}.
%We also subconsciously anchor on particularly salient sensory inputs that become representative of that location for us~\cite{Weisberg2016, Miller2013, Keatley2021}, and it is those anchor points that we most strongly recall into the future.}
%



%Taking inspiration from the way humans recall spatial information~\cite{Helbing2020, Oliveira2016, Weisberg2016}, \nrscomment{update citations} we present \emph{COMPASS}, an approach for spatial pattern match querying that... \nrscomment{summarize high level contribution}. 



% 
%\par{Think about times someone has described a location they've visited to you.
%Does \textit{"it is 1.73 miles from the center of the winery to the furthest edge of the property. 500 yards to the southwest of the cellar door is a duck pond. 300 yards north of that pond is a statue which is 400 yards west-south-west of the cellar door"} sound realistic?
%What about \textit{"I remember standing on the balcony of the cellar door and away in the distance to my left was a beautiful duck pond, and off to the right of that was this very odd statue of a goat in a top hat."}?
%Rather than focusing on how a map sees the world, we focus on how humans see the world.}
%


%Traditionally, analysts will get to a \textit{near-enough} start point and then begin the excruciating manual review of remote sensing imagery, photography and other reports to try to match the objects they know about to the location they are searching for in the geospatial configuration they are expecting.

%is searching for a specific location, they have many tools at their disposal. If they remember the exact name of the location they are looking for, most modern systems can return an exact match. If they know the physical address of the location, they can use that information to determine what they are looking for. There are situations, however, where a user has access to imperfect information about a location they are searching for. 

%%Perhaps it is a location they visited a long time ago which has faded in memory or a vague recommendation from a friend. Perhaps they attempt to combine fragmented evidence for a police investigation or fuse intelligence in a military operation. %%NSCH add as motivation somewhere?

%Regardless of the reason, common approaches to the problem now see users employing GIS tools for their bold adjust to get them to the correct general area and then rely on a manual last-mile effort typically involving the visual inspection of remote sensing imagery,searching for distinct landmarks or terrain features that match their partial information. While the most important, the last mile effort is also the bottleneck.

%The two key research questions driving the development of \textit{GESTALT} are:
%\begin{enumerate}
%\item What components of the \textit{last-mile} search can be automated
%\item How helpful is automation in completing the \textit{last-mile} search?
%\end{enumerate}

%\subsection{Overview of Running-Example}
%A running scenario illustrates the architecture in use throughout the \textit{GESTALT} paper. 
%The running scenario is a \textit{last-mile search} problem. The searcher's task is to find out which of the wineries they visited before attending a lunch with %friends to share their recommendations. 
%However, because of over-indulgence while on tour, they are struggling to recall which wineries they visited.
%They know that the general region is the Swan Valley Wine Region near Perth in Western Australia. There are more than 30 wineries in the small Swan Valley Wine %Region and far more locations, including the many distilleries, breweries, chocolate shops and other auxiliary venues. 
%They remember a few key facts about their trip, but the details are fuzzy. That is, they have \textit{partial information} about each of the wineries:
%\begin{enumerate}
%	\item They have a photo in front of Oakover Grounds, so know they must have visited it. 
%	\item They recall drinking a lovely Verdehlo at a table made from wine barrels. 
%	\item They didn't like the Chenin Blanc they were drinking in the place with palm trees.
%	\item There was a noisy generator at one that ruined the ambience. 
%\end{enumerate}	
%They could find Oakover Estate quickly enough using Google Maps but didn't recognise any other names. 
%Searching for "Winery with wine barrel tables" yielded inconclusive results listing most of the wineries from the original set. The lack of search-space reduction is likely due to the text associated with most wineries mentioning wine, barrels and tables. 
%Ordinarily, they would manually use google maps satellite imagery, street view and user review photos to identify which wineries they were at. Indeed, there is a better way.

%\subsection{Contributions}
%The critical contribution of this work is the architecture for \textit{GESTALT}, an end-to-end pipeline for extracting geospatial data, transforming it into coherent object-location relations, and loading it into a store and searching it. Further, this work evaluates the performance of popular clustering algorithms on the geospatial object ownership assignment inference task. It contributes a new gold standard \textit{Swan Valley Wineries} dataset and a proof of concept implementation of the architecture using the Wineries Dataset.This paper will explore the sourcing and creation of datasets for the \textit{GESTALT} project in section \ref{section:datasets}. It will then explore the architecture and define the goals for each subsystem in section \ref{section:architecture} before exploring the implementation in section \ref{section:implementation}. A detailed comparison of Object ownership assignment methods is in section \ref{section:results} before exploring related work in section \ref{section:related}. The paper concludes by identifying future work in section \ref{section:conclusion}.  

%\subsection{Organization}
%This paper will explore the sourcing and creation of datasets for the \textit{GESTALT} project in section \ref{section:datasets}. It will then explore the architecture and define the goals for each subsystem in section \ref{section:architecture} before exploring the implementation in section \ref{section:implementation}. A detailed comparison of Object ownership assignment methods is in section \ref{section:results} before exploring related work in section \ref{section:related}. The paper concludes by identifying future work in section \ref{section:conclusion}.  

%that identifying locations based on collections of objects associated with them is a technique actively used by investigators. 
%The investigative requirements are evidently great enough to necessitate Bellingcat developing an \textit{OpenStreetMap Search Tool} to search for locations using objects\footnote{\href{https://www.bellingcat.com/resources/how-tos/2023/05/08/finding-geolocation-leads-with-bellingcats-openstreetmap-search-tool/}{https://www.bellingcat.com/resources/how-tos/2023/05/08/finding-geolocation-leads-with-bellingcats-openstreetmap-search-tool/}}.
%Bellingcat's tool puts a user-friendly interface on the OSM Overpass-Turbo Interface\footnote{\href{https://overpass-turbo.eu/}{https://overpass-turbo.eu/}} 
%Our testing indicates a rapid degradation in query execution time as scale increases, and is distinct from our work in that it only queries OSM data, only checks for set intersections of objects within a given proximity threshold of each other and does not allow for pictorial querying.
%\emph{GESTALT} approaches the same problem in a more comprehensive way, aiming to help investigators with their \textit{last-mile} search to identify locations of interest based on the objects that they know (or suspect) are associated with the location. 
