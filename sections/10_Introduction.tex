\section{Introduction}

\label{section:introduction}

Modern spatial search methods are derived from traditional search techniques that were not designed with spatial search constraints in mind.
Keyword querying, which is a natural constraint in relational data, can be applied to narrow spatial regions, but the full richness of the spatial information is lost~\cite{Guo2015, Cao2012, Zhang2009, Osul2023}.
Likewise, simple metric or topological constraints like "near", "within x miles of", etc. cannot capture the inherent complexity of spatial data, where every entity relates to every other entity through its position in a region~\cite{Carniel2020,Bertella2022,Carniel2023}. 
Current methods for searching spatial data require encoding schemes that collapse rich spatial information down to basic relationships that fail to capture the most important aspect of the data - how multiple spatial entities relate to each other directionally.

Under this paradigm, natural spatial pattern queries like "Where is there a pond, a statue, and a bridge arranged in a triangle?" are impossible to answer.
Previous work in spatial pattern matching, which handles exactly this type of multi-entity directional query, requires encoding the data in densely connected graphs or multigraphs, or segmenting space with respect to every object individually~\cite{Folkers2000, Fang2019,Chen2019,Fang2019}.
Because of their underlying encoding schemes, these approaches are intractable for any reasonable number of objects and query constraints.
This gap between the natural spatial expressiveness of directional queries and the intractibility of resolving them for nontrivially-sized collections of spatial data is the reason modern search engines do not support them.

We present \textit{\textbf{C}ardinal \textbf{O}rientation \textbf{M}anipulation and \textbf{P}attern-\textbf{A}ware \textbf{S}patial \textbf{S}earch}, (\emph{COMPASS})\footnote{\url{https://github.com/osullik/COMPASS_GEOSEARCH}}, a suite of data structures and algorithms that begin to address the scalability limitations of directional spatial pattern matching.
To accomplish this, we encode the directional relationships between objects in abstracted matrix representations rather than the graph-based structures that most current approaches use.
We enable search over these matrix representations through a series of recursive spatial pattern-matching algorithms that traverse the objects, searching \emph{if} there exists at least one set of objects that satisfies the constraints (rather than retreiving all sets that do).
This simplification of the problem allows us to achieve worst case complexity of $\mathcal{O}(n^2)$ in the number of objects, a significant improvement over previous methods.
By simplifying the problem, we enable users to search over geospatial objects using directional constraints at scale, which has not been achieved previously.

Further, when objects are grouped semantically at a `location' the \emph{if} question can be used to retrieve any locations that contain at least one set of objects matching the query constraints.
For instance, "Find all parks with a pond, a statue, and a bridge arranged in a triangle" becomes answerable.
To encourage evaluation of future methods against our initial matrix-based solution, we provide a data generator that produces location and object databases and pictorially-specified directional spatial queries to run against them.

We begin the rest of this paper by describing relevant background on spatial search in section \ref{section:background}. Sections \ref{section:data_structures} and \ref{section:methods} outline our proposed solution. We evaluate each algorithm's theoretical complexity and efficiency on the datasets we generate in sections \ref{section:experimental_setup} and \ref{section:results} and briefly summarize relevant related work in section \ref{section:related} before concluding.