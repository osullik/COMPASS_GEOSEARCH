\section{Conclusion}
\label{section:conclusion}
\normalsize
We present \emph{COMPASS}, a collection of data structures and algorithms that enable search over directional spatial relations at scale. 
To accomplish this, we develop the matrix-based \textit{concept map} data structure to implicitly encode the otherwise unmanageable number of pairwise directional relations between objects.
We further simplify the previously intractable directional spatial pattern matching problem by encoding geospatial objects hierarchically, dividing them into semantically meaningful groups called \textit{locations}. 
We provide a detailed comparison of existing approaches to spatial pattern matching, and show that the theoretical worst-case complexity of our approach is significantly better than any previous approaches that handle directional relations.
We hope that by taking the first step in solving this simplified version of the directional spatial pattern matching problem, this work will inspire further attempts to break the complexity barrier of this powerful but computationally challenging method of spatial search.



 

 











%Ownership assignment is implemented in Python in two ways. The first is the trivial implementation, where the location returned from the OSM query is a bounding polygon. In this case, if the point lies within the minimal enclosing rectangle of the polygon, it is added as a 'member' of that location. 