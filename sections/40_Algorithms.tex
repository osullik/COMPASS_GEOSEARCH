\section{COMPASS Search Algorithms}
\label{section:methods}
Our choice of data structures allows us to employ search methods that would otherwise be unsuitable for graph structures. 
This section outlines the methods used to accomplish \emph{COMPASS}'s three forms of spatial search.

\begin{algorithm}[h!]
    \caption{Location-Object Search}\label{alg:locSearch}
    \begin{algorithmic}
        \State{\textit{\textbf{Q} A query of Objects with names and coordinates}}
        \State{\textit{\textbf{D} A database of Objects with names and coordinates}}
        \State{\textit{\textbf{Q.Loc} Midpoint of query canvas, representing position of Location sought}}
        \State{\textit{\textbf{D.Loc} Location to which Objects in D are assigned}}
        \State{- - - - -}
        \Procedure{objectLocationSearch}{$Q$, $D$, $Q.Loc$, $D.Loc$}             \State{$D.Dict \leftarrow$ makeLocationCentricStructure($D$,$D.Loc$)}
            \State{$Q.Dict \leftarrow$ makeLocationCentricStructure($Q$,$Q.Loc$)}
            \State{$matches \leftarrow$ 0}
            \For{Each $quadrant$ in [$NW$, $NE$, $SW$, $SE$]}
                \For{Each point $P$ in $Q.Dict[quadrant]$}
                    \If{$D.Dict[quadrant]$ contains $P$}
                        \State{$matches \leftarrow matches$ + 1} 
                        \State{Pop $P$ from $D.Dict[quadrant]$}                     \EndIf
                \EndFor
            \EndFor
            \State{\textbf{return} $matches$} \Comment Used to rank against other candidate locs
        \EndProcedure
        \State{--------------------------------------------------------------------------------------------------------}
        \State{\textit{\textbf{D} A database of Objects with names and coordinates}}
        \State{\textit{\textbf{D.Loc} Location to which Objects in D are assigned}}
        \State{- - - - -}
        \Procedure{MakeLocationCentricStructure}{$D$, $D.Loc$}
            \State{$Dict \leftarrow $\{ $NW$: [ ], $NE$:[ ], $SW$: [ ], $SE$: [ ] \}}
            \For{$Obj$ in $D$}                \If{$Obj.y$ < $D.Loc.y$ and $Obj.x$ < $D.Loc.x$}
                                            \State{Add $Obj.Name$ to $Dict.SW$}
                \ElsIf{$Obj.y$ < $D.Loc.y$ and $Obj.x$ >= $D.Loc.x$}
                        \State{Add $Obj.Name$ to $Dict.SE$}
                                    \ElsIf{$Obj.y$ >= $D.Loc.y$ and $Obj.x$ < $D.Loc.x$}
                                            \State{Add $Obj.Name$ to $Dict.NW$}
                \Else
                        \State{Add $Obj.Name$ to $Dict.NE$}
                                    \EndIf 
            \EndFor
            \State{\textbf{return} $Dict$}
        \EndProcedure
    \end{algorithmic}
\end{algorithm}


\subsection{Location-Object Search}
\par{
    \textit{Location-Object} search allows a user to retrieve locations by specifying directional constraints for objects relative to the center of the location that `owns' them. 
    It assumes a globally aligned coordinate system and the existence of a \textit{Location-Object} data structure for each location in the region of interest.
    To issue a \textit{Location-Object} query, a user inserts query objects around the `center' of the query canvas, encoding directional constraints relative to the desired location (NW/NE/SW/SE). 
    \textit{Location-Object} search is effectively a set intersection operation. 
    It proceeds by iterating through the candidate locations, searching each location's (NW/NE/SW/SE) quadrants for object-quadrant relationships that match the directional constraints specified in the query.
    Figure \ref{fig:CM-LO-Query} shows how to query the Location-Object data structure, and the process is formalized by Algorithm~\ref{alg:locSearch}.
    The search function returns candidate locations ranked by the number of query terms that match the appropriate quadrant for the location and can return fuzzy partial matches.  
    



\begin{algorithm}[h]
    \caption{Concept Mapping}\label{alg:geoToGrid}
    \begin{algorithmic}
        \State{\textit{\textbf{D} A database of Objects with names and coordinates}}
        \State{- - - - -}
        \Procedure{createConceptMap}{$D$}
            \State{xList, yList $\leftarrow$ []} 
                        \State{$M$ $\leftarrow$ [[][]]} \Comment{A matrix of 0s}
                        \For{$Obj$ in $D$} \Comment{Where $D$ sorted (desc) on y coord}     
                \State{Add $Obj$ to $yList$}             \EndFor
                        \For{$Obj$ in $D$}  \Comment{Where $D$ sorted (desc) on x coord}  
                \State{Add $Obj$ to $xList$}
            \EndFor          
            \For{$Obj$ in $xlist$}          
                \State{$i \leftarrow$ index of $Obj$ in $xList$}
                \State{$j \leftarrow$ index of $Obj$ in $yList$}               
                \State{$M$[$i$][$j$] $\leftarrow$ $Obj.Name$}             \EndFor        
            \State{\textbf{return} $xList$, $yList$}
        \EndProcedure

        \State{--------------------------------------------------------------------------------------------------------}
        \State{\textit{\textbf{xList} A list of terms sorted by x-coordinate}}\Comment{Same number of terms as $yList$}
        \State{\textit{\textbf{yList} A list of terms sorted by y-coordinate}}\Comment{Same number of terms as $xList$}
        \State{- - - - -}
        \Procedure{Order Search Terms}{$xList$, $yList$}
            \State{$Traversed$ $\leftarrow$ []}
            \State{$dir$ = 'x'}
            \While{len($Traversed$) < len($xList$)} 
                \If{$xList$ not empty and $dir$ == 'x'}
                    \State{$term$ = pop($xList$)}
                    \If{$term$ not in $Traversed$}
                        \State{Append $term$ to $Traversed$}
                        \State{$dir$ = 'y'}
                    \EndIf
                \EndIf
                \If{$yList$ not empty and $dir$ == 'y'}
                    \State{$term$ = pop($yList$)}
                    \If{$term$ not in $Traversed$}
                        \State{Append $term$ to $Traversed$}
                        \State{$dir$ = 'x'}
                    \EndIf
                \EndIf
            \EndWhile
            \State{\textbf{return} $Traversed$}
        \EndProcedure
    \end{algorithmic}
\end{algorithm}

\subsection{Object-Object Search}

\textit{Object-Object} search allows a user to retrieve locations by specifying directional constraints between objects and comparing them to the directional relationships of objects in the underlying database. 
It assumes compliance with a globally aligned coordinate system.
The \textit{Object-Object} query is specified pictorially and is converted into a concept map matrix to capture its directional constraints.
The \textit{Recursive Grid Search} (RGS) (Algorithm~\ref{alg:gridSearch}) uses the input concept map query to identify which locations have objects matching the directional constraints specified in the query. 
The search process is depicted in figure \ref{fig:CM-OO-Query}, but in general terms, the \textit{Recursive Grid Search} traverses the database objects starting from the north.
Once the northernmost query object is encountered, any database objects previously encountered are irrelevant to the query and can be pruned.
The same holds for the western direction.
Pruning occurs recursively in alternating directions until all query objects are found or the traversal completes without matching all query terms.


\begin{algorithm}[h!]
    \caption{Object-Object Search}\label{alg:gridSearch}
    \begin{algorithmic}
        \State{\textit{\textbf{Q} A query of Objects with names and coordinates}}
        \State{\textit{\textbf{D} A database of Objects with names and coordinates}}
        \State{- - - - -}
        \Procedure{objectObjectSearch}{$Q$, $D$} 
            \State{$D.M$ $\leftarrow$ createConceptMap($D$)}
            \State{$Q.M$ $\leftarrow$ createConceptMap($Q$)}
            \State{$Q.L$ $\leftarrow$ orderSearchTerms($M$)}
            \State{\textbf{return recursiveConceptMapSearch}($D.M$,$Q.L$)}
        \EndProcedure
        
        \State{--------------------------------------------------------------------------------------------------------} 
        \State{\textit{\textbf{D.M} A ConceptMap Matrix with objects or 0s; [0,0] is NW most point}}
        \State{\textit{\textbf{Q.L} A NW to SE ordered list of query objects}}
        \State{\textit{\textbf{Dir} The alternating cardinal direction to prune from, 'N' (default) or 'W'}}
        \State{- - - - -}
        \Procedure{recursiveConceptMapSearch}{$D.M$,$Q.L$,$Dir$}             \If{$Q.L$ has only 1 item}\Comment{Base Case}
                \If{pop($Q.L$) in $D.M$}
                    \State{\textbf{return} $True$}              
                \Else
                    \State{\textbf{return} $False$}
                \EndIf            
            \EndIf
            \State{$P\leftarrow$ pop($Q.L$)}                
            \If{$P$ not in $D.M$}
                \State{\textbf{return} $False$}               
            \EndIf
            \State{$D.M^\prime \leftarrow$ Prune all objects $Dir$ of $P$}
            \State{$Dir \leftarrow$ \textbf{changeDirection}($Dir$)}
            \State{\textbf{return recursiveConceptMapSearch}($D.M^\prime$,$Q.L$,$Dir$)}
        \EndProcedure
        \State{}
        \Procedure{changeDirection}{$Dir$}
        \If{$Dir$ == 'N'}
            \State{\textbf{return} 'W'}
        \Else
            \State{\textbf{return} 'N'}
        \EndIf
        \EndProcedure
    \end{algorithmic}
\end{algorithm}

\begin{figure}[h]
    \centering
        \includesvg[width=0.5\textwidth]{figures/rot-inv-diagram.svg}
    \caption{\textbf{Cardinality-Invariant Object-Object search Method. The query configuration is rotated at most 3$Q$ times (where $Q$ is the number of query terms) to align each term to the North, West, and Northwest reference points. The original query configuration (left) is Pictured above, followed by two of the nine possible rotations for these three query terms.}}\label{figure:rot-inv} 
\end{figure}

\begin{algorithm}[h]
    \caption{Cardinality-Invariant Object-Object Search}\label{alg:cardInvSearch}
    \begin{algorithmic}
        \State{\textit{\textbf{Q} A query of Objects with names and coordinates}}
        \State{\textit{\textbf{D} A database of Objects with names and coordinates}}
        \State{- - - - -}
        \Procedure{cardinalityInvariantObjectObjectSearch}{$Q$, $D$}             \State{$B \leftarrow$ Bounding box of points in $Q$}
            \State{$C\leftarrow$ centroid of points in $Q$} 
            \State{$N\leftarrow$ North Center point of $B$} 
            \State{$W\leftarrow$ West Center point of $B$} 
            \State{$NW\leftarrow$ Northwest Center point of $B$} 
            \State{$D.M$ $\leftarrow$ createConceptMap($D$)}
            \For{Each $Dir$ in [$N, W, NW$]} \Comment{Reference \textbf{Dir}ection}
                \For{Each point $P$ in $Q$}
                    \State{$\theta$ $\leftarrow$ $\angle$ $R$$C$$P$ }\Comment{Measured clockwise}
                    \State{$Q^\prime \leftarrow$ Rotate($Q$, $\theta$)}\Comment{Rotate counterclockwise}
                    \State{$M$ $\leftarrow$ createConceptMap($Q^\prime$)}
                    \State{$Q^\prime.L$ $\leftarrow$ orderSearchTerms($M$)}
                    \If{recursiveObjectObjectSearch($D.M$, $Q^\prime.L$) == True}
                        \State{\textbf{return} True}               
                    \EndIf
                \EndFor
            \EndFor
            \State{\textbf{return} False}
        \EndProcedure
    \end{algorithmic}
\end{algorithm}


\subsection{Cardinality-Invariant Object-Object Search}
\textit{Cardinality Invariance} extends \textit{Object-Object} search (Algorithm~\ref{alg:gridSearch}) by removing the assumption of compliance with a globally aligned coordinate system.
We address the coordinate system alignment issue by rotating the query pattern the minimal number of times needed to cover the distinct configurations relative to the actual cardinal directions.
Algortihm~\ref{alg:cardInvSearch} describes the process of rotating the points of the query term around their centroid, and a visual depiction is in figure~\ref{figure:rot-inv}.
We generate the minimum bounding rectangle for the query points and impose three reference points: `North', `West', and `Northwest' on the bounding rectangle relative to the centroid. 
For each point in the query, we compute the angle between it, the query centroid, and each reference point. 
Rotating the query configuration by each angle aligns the given point to each reference direction.
The unique set of query configurations obtained in this manner represents the minimum set of object-object queries required to determine if the pattern specified in the query matches, irrespective of cardinality. 


        }

 







