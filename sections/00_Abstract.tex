%Humans spend a lot of time searching for things. 
%With the advent of tools like google maps and open street maps, people can search through geospatial data at a whim. 
%These tools focus on providing exact matches to queries or a list of candidate locations based on the user's query. 
%Frequently, searchers only have access to partial information. 
%Whether it has been a long time since visiting a location, they have a vague recommendation from a friend or are an investigator trying to identify a location to solve a crime- a common problem is how to find a location of interest based on partial information. 
%This project designs \textit{the \textbf{G}eospatially \textbf{E}nhanced \textbf{S}earch with \textbf{T}errain \textbf{A}ugmented \textbf{L}ocation \textbf{T}argeting (\textbf{GESTALT})}, and implements a proof-of-concept of the proposed architecture. 
%Based on a new best-case dataset developed for this project, \textit{The Swan Valley Wineries dataset}, demonstrates the functionality and utility of \textit{GESTALT} while identifying substantial opportunities for future work. 


%Geographic information systems (GIS) provide users with a means to efficiently search over spatial data given certain key pieces of information, like the coordinates or exact name of a location of interest. 
Effective spatial search typically requires users to know the exact coordinates or name of the location they seek.
When only partial or imperfect information is available, distance-based nearest neighbor searching can be used.
However, that requires users to estimate distance between objects in the environment, a task that humans have been shown to perform poorly at.
The more natural spatial search alternative is pictorial querying, where a sketch map is used to enable users to place objects in a configuration where their relative positions define the query constraints.
Most approaches to pictorial query searching are intractable for any realistic number of query and database terms.
We present \emph{COMPASS}, a collection of data structures and algorithms that address the scalability issues that previously rendered pictorial-based searching infeasible.


%However, current GIS capabilities do not enable users to easily search for locations about which they have imperfect or incomplete information. 
%\nrscomment{Motivation of pictorial querying}

%\nrscomment{Issues of tractability}

%\nrscomment{How we reframe the problem and solve it tractably}