\section{Methods}
\label{section:methods}

\subsection{Object-Location Search}
\textbf{Object-location} relations are encoded as a per-location index that subdivides the search space into four quadrants - NorthWest, NorthEast, SouthWest, and SouthEast, splitting on the location coordinates. The division of space aims to emulate how people relate the world to their position in it geospatially. 
\nrscomment{describe more here}
Searching the object-location structure involves a simple iteration through the candidate locations, searching each one's NW/NE/SW/SE quadrants for the relevant query objects, and ranking the locations based on how many objects were found in the correct quadrant.



\nrscomment{this needs to be written based on how it's implemented.}

\nrscomment{Generalize makeLocationCentricStructure to use it for query and data, then do set intersections to get num matches?}

\nrscomment{Return ranked set of locations as output}

\begin{algorithm}[H]
    \caption{Location-Object Search}\label{alg:locSearch}
    \begin{algorithmic}
        \State{\textit{\textbf{Q} A pictorial query consisting of (x,y) coordinates}}
        \State{\textit{\textbf{$C$} Midpoint of query canvas, representing position of Loc sought}}
        \State{- - - - -}
        \Procedure{objectLocationSearch}{$Q$} %\Comment{$M$ a matrix of a Loc's objects, $L$ an ordered list of objects to search for, $D$ is the pruning direction}
            \State{$D \leftarrow$ makeLocationCentricStructure($T$,$X$)}  \nrscomment{how to get T and X}
            \State{$matches \leftarrow$ 0}
            \For{Each point $P$ in $Q$}
                \If{$D$ contains $P$ matching in same quadrant as $Q$}
                    \State{$matches \leftarrow matches$ + 1}               
                \EndIf
            \EndFor
            \State{\textbf{return} $matches$}
        \EndProcedure
        \State{}
        \State{\textit{\textbf{T} a Location Table with ID, Name, Lat \& Long of its objects}}
        \State{\textit{\textbf{Loc} a (x,y) tuple of the the location}}
        \State{- - - - -}
        \Procedure{MakeLocationCentricStructure}{$T$,$Loc$}
            \State{$D \leftarrow $\{$NW$:[], $NE$:[], $SW$:[], $SE$:[]\}}\Comment{Dictionary}
            \For{$R$ in $T$}\Comment{R is a row describing a single object}
                \If{$R.y$ < $Loc.y$}
                    \If{$R.x$ <= $Loc.x$}
                        \State{Add $R.Name$ to $D.SW$}
                    \Else
                        \State{Add $R.Name$ to $D.SE$}
                    \EndIf
                \Else
                    \If{$R.x$ <= $Loc.x$} 
                        \State{Add $R.Name$ to $D.NW$}
                    \Else
                        \State{Add $R.Name$ to $D.NE$}
                    \EndIf
                \EndIf 
            \EndFor
            \State{\textbf{return} $D$}
        \EndProcedure
    \end{algorithmic}
\end{algorithm}

\subsection{Object-Object Search}
\textbf{Object-object} relations encode objects in sparse $NxN$ matrices, where $N$ is the number of objects at a location. 
Using algorithm \ref{alg:geoToGrid}, we assign each object in a given location to a position $(i,j)$ in the matrix where $i$ is its order of appearance from north to south and $j$ is the same object's order of appearance from west to east (depicted in Figure \ref{fig:CM-OO-Setup}). \nrscomment{clean up this subsection/ describe more}

Searching the object-object concept maps involves using the recursive grid search (Algorithm \ref{alg:gridSearch}) to identify which locations have objects matching the configuration specified in the pictorial query, and returning the relevant set of results. 
The pictorial query is first converted into a concept map matrix using the process outlined in Appendix \ref{appendix:conceptcreation} and the search terms in the matrix are ordered into a queue with the northernmost enqueued first, followed by the westernmost, then the second northernmost, and so on until all search terms are in the queue.
Then, the grid search is employed to recursively prune the search space and quickly eliminate candidate locations that do not match the query configuration.
This process is depicted visually in Figure \ref{figure:ConceptMap}.



\begin{algorithm}[H]
    \caption{Concept Mapping}\label{alg:geoToGrid}
    \begin{algorithmic}
        \State{\textit{\textbf{T} a Location Table with ID, Name, Lat \& Long of its objects}}
        \State{- - - - -}
        \Procedure{createConceptMap}{$T$}
            \State{LonList, LatList $\leftarrow$ []} 
            %\State{LatList = []} \Comment{A list}
            \State{$M$ $\leftarrow$ [[][]]} \Comment{A matrix of 0s}
            %\State{Sort rows from North to South}
            \For{$R$ in $T$} \Comment{Where $T$ sorted North to South}     
                \State{Add R.$ID$ to $LatList$} \Comment{$R.ID$ is the obj unique ID}
            \EndFor
            %\State{Sort rows from West to East}         
            \For{$R$ in $T$}  \Comment{Where $T$ sorted West to East}  
                \State{Add R.$ID$ to $LonList$}
            \EndFor          
            \For{$ID$ in $Lonlist$}          
                \State{$i \leftarrow$ index of $ID$ in $LonList$}
                \State{$j \leftarrow$ index of $ID$ in $LatList$}               
                \State{$M$[$i$][$j$] $\leftarrow$ $ID.Name$} \Comment{Name of object with $ID$}
            \EndFor        
            \State{\textbf{return} $M$} \Comment{Having $[0,0]$ as NW corner}
        \EndProcedure

        \State{}
        \State{\textit{\textbf{M} A ConceptMap Matrix with objects or 0s}}
        \State{- - - - -}
        \Procedure{Order Search Terms}{$M$}
            \If{$M$ has single Item}
                \State{Return $M$}
            \EndIf
            \State{$Traversed$ $\leftarrow$ []}
            %\State{$R$, $C$ = Number of rows and columns}
            %\State{$Rd$ = ($R$+$C$)-1}
            \For{$r$ in range ($R$+$C$)-1} \Comment{$R$, $C$ = Number of rows and columns}
                \For{$i$ in range $r$+1}
                    %\State{$j$ = $r$-$i$}
                    \If{$i$ < $R$ and $r$-$i$ < $C$ and $M$[$i$][$r$-$i$] != 0}
                        %\If{$M$[$i$][$j$] != 0}
                            \State{Add $M$[$i$][$r$-$i$] to $Traversed$}
                        %\EndIf
                    \EndIf 
                \EndFor
            \EndFor
            \State{\textbf{return} $Traversed$}
        \EndProcedure
    \end{algorithmic}
\end{algorithm}




\begin{algorithm}[H]
    \caption{Object-Object Search}\label{alg:gridSearch}
    \begin{algorithmic}
        \State{\textit{\textbf{M} A ConceptMap Matrix with objects or 0s; [0,0] is NW most point}}
        \State{\textit{\textbf{L} A NW to SE ordered list of objects to search for}}
        \State{\textit{\textbf{Dir} The alternating cardinal direction we prune from, 'N' (default) or 'W'}}
        \State{- - - - -}
        \Procedure{recursiveObjectObjectSearch}{$M$,$L$,$Dir$} %\Comment{$M$ a matrix of a Loc's objects, $L$ an ordered list of objects to search for, $D$ is the pruning direction}
            \If{$L$ has only 1 item}\Comment{Base Case}
                \If{pop($L$) in $M$}
                    \State{\textbf{return} $True$}              
                \Else
                    \State{\textbf{return} $False$}
                \EndIf            
            \EndIf
            \State{$P\leftarrow$ pop($L$)}                
            \If{$P$ not in $M$}
                \State{\textbf{return} $False$}               
            \EndIf
            \State{$M^\prime \leftarrow$ Prune all objects $Dir$ of $P$}
            \State{$Dir \leftarrow$ \textbf{changeDirection}($Dir$)}
            \State{\textbf{return recursiveObjectObjectSearch}($M^\prime$,$L$,$Dir$)}
        \EndProcedure
        \State{}
        \Procedure{changeDirection}{$Dir$}
        \If{$Dir$ == 'N'}
            \State{\textbf{return} 'W'}
        \Else
            \State{\textbf{return} 'N'}
        \EndIf
        \EndProcedure
    \end{algorithmic}
\end{algorithm}




\begin{algorithm}[H]
    \caption{Cardinality-Invariant Object-Object Search}\label{alg:cardInvSearch}
    \begin{algorithmic}
        \State{\textit{\textbf{Q} A pictorial query consisting of (x,y) coordinates}}
        \State{- - - - -}
        \Procedure{objectObjectSearch}{$Q$} %\Comment{$M$ a matrix of a Loc's objects, $L$ an ordered list of objects to search for, $D$ is the pruning direction}
            \State{$B \leftarrow$ Bounding box of points in $Q$}
            \State{$C\leftarrow$ centroid of points in $Q$} 
            \State{$N\leftarrow$ North Center point of $B$} 
            \State{$W\leftarrow$ West Center point of $B$} 
            \State{$NW\leftarrow$ Northwest Center point of $B$} 
            \For{Each $Dir$ in [$N, W, NW$]} \Comment{Reference \textbf{Dir}ection}
                \For{Each point $P$ in $Q$}
                    \State{$\theta$ $\leftarrow$ $\angle$ $R$$C$$P$ }\Comment{Measured clockwise}
                    \State{$Q' \leftarrow$ Rotate($Q$, $\theta$)}\Comment{Rotate counterclockwise}
                    \State{$M$ $\leftarrow$ createConceptMap($Q'$)}
                    \State{$L$ $\leftarrow$ orderSearchTerms($M$)}
                    \If{recursiveObjectObjectSearch($M$,$L$) == True}
                        \State{\textbf{return} True}               
                    \EndIf
                \EndFor
            \EndFor
            \State{\textbf{return} False}
        \EndProcedure
    \end{algorithmic}
\end{algorithm}