\section{Related Work}
\label{section:related}
\normalsize




\par{
    Most practical applications of spatial search aim to answer queries like "Where are the Italian restaurants near me?".
    Most of the work in this area is based on Spatial Keyword Queries (SKQ), which handle topological and metric constraints but lack support for directional constraints~\cite{Guo2015, Cao2012, Zhang2009, Osul2023}, or conventional spatial joins, which do not typically capture more than two-way directional relations~\cite{Jacox2007}. 
    Directional constraints are densely interrelated, and current methods for encoding and searching over them (set intersection, subgraph matching, and constraint satisfaction problems) are impractical at scale.
    We describe the notable works in these areas and include a complexity analysis in Table \ref{Table:related_work}.
}


\subsection{Set Intersection (SI)}
\par{  
    Set-intersection (SI) approaches allow users to specify a combination of keyword, topological, metric, and directional constraints to query against a database of objects~\cite{DiLoreto1996, Soffer1996, Soffer1997, Soffer1998a, Soffer1999}.
    These approaches are limited by the requirement to extract the sets matching each specified constraint from the underlying representation before intersecting them. 
    Most modern approaches employ set intersection as an initial keyword filter before conducting the more expensive search operations described below~\cite{Schwering2014, Osul2023}.
    

\subsection{Subgraph Matching (SGM)}
\par{
    Subgraph matching (SGM) approaches assume the database is encoded as a graph of objects and relations, and queries are encoded as a subgraph of objects and constraints of interest.
    These methods then seek to identify where the query pattern exists in the database graph. 
    Spatial SGM methods like PQIS~\cite{Folkers2000} and Spacekey$_{MPJ}$~\cite{Fang2019} approach exponential in the number of constraints, and ESPM and SpaceKey$_{MSJ}$ are bound by the number of nodes in exponential~\cite{Chen2019} or quartic~\cite{Fang2019} time.
    Fuzziness and partial matching adds further complexity~\cite{Fang2019}.
    }
    
\subsection{Constraint Satisfaction Problems (CSP)}
    \par{
    Constraint Satisfaction Problems (CSP) for spatial pattern matching find an assignment of valid variables (database objects) given the query constraints.
    In the worst case, where constraints are numerous, specificity is low, and constraints are poorly ordered, CSP approaches perform poorly.
   % 
    Most CSP are at least cubic in the number of objects~\cite{Dylla2017}.
    Some approaches like StarVars, are exponential in the number of objects~\cite{Lee2013}.
    \textit{Forward Checking} algorithms, like MSJ$_{WR}$, MSJ$_{JWR}$, and $_{MSJ}$ are exponential in the number of constraints or in the number of query terms~\cite{Papadias1998}.
    \textit{Sketchmapia,} which combines a qualitative constraint network representation with a subgraph matching solver, we estimate performs similarly~\cite{Schwering2014, Jan2015}.
   

\subsection{Summary}

\par{
    %All of the current approaches for spatial pattern matching are unsuitable for the directionally constrained search task under consideration.% \emph{COMPASS} to fulfill. 
    Table \ref{Table:related_work} shows that \emph{COMPASS} offers a combination of directional constraints, cardinality invariance and better worst-case complexity than systems with comparable functionality.
    Formulating spatial pattern matching problems as graph or constraint network problems has fostered search methods dependent on those data structures. 
    Set intersection, subgraph matching, and constraint satisfaction problems all degrade when the number of constraints and query terms increase, making them unsuitable for dense directional relations. 
    The data structures and algorithms presented in \emph{COMPASS} re-frame the underlying representation of spatial pattern matching problems so that new search methods can be applied that scale reasonably for directional relation-based search.
}

%%% JOURNALL
% \subsection{Human Spatial Cognition and Reasoning}
% \par{
%     Research in human spatial reasoning shows that humans perceive the world not as a map but as a collection of sensory anchors around salient locations. 
%     %
%     Schwering et al.~\cite{Schwering2014} assert that the three most common sources of error in how humans recall their environment are \textit{errors in direction, errors in distance} and \textit{errors in schematic structure}.
%     %
%     Humans are rarely aware of the cardinal directions in an environment, perform poorly when estimating distances between objects, and inherently view the world as relative to their place in it.
%     }
% \par{
%     Errors of \textit{direction} and \textit{distance} are easily understood as a consequence of humans favoring a local (rather than global) approach to direction finding and the well-documented difficulty of estimating distances of unknown objects at uncertain ranges in varying lighting and weather conditions. 
%     %
%     The errors of \textit{schematic structure} partly occur because people tend to anchor memories on salient objects they encounter while experiencing a location~\cite{Helbing2020}, so they recall those objects more readily than others.
% }
% \par{
%     The human preference for a local worldview that anchors on the relative positioning of salient objects motivates us to use \textit{directional} relations between spatial objects as the core of our spatial pattern matching approach.
% }

% \subsubsection{Other Problem Formulations}
% \par{
%     Other approaches do not fit neatly into either SGM or CSP. For example, Liu et. al. developed an Open Shape-based Strategy (OSS) to search for spatial objects based on query terms that are not aligned with the global coordinate system~\cite{Liu2003}.
%     Della Penna et. al. develop a framework for implicitly and explicitly encoding relative spatial positions, and the Spatial Relation Query Language (SRQL).  
%     SRQL is a SQL-like language that allows users to search through documents for elements arranged in certain positions like \textit{left, right, above} or \textit{below}~\cite{Dellapenna2012, Dellapenna2017}
% }



%They emphasize that the essence of the qualitative spatial reasoning approach is to abstract the infinite complexity of spatial and temporal configurations in the real world to a solvable representation. 

%As the system's ability to be more precise and powerful in its query formulation increased, so does the number of constraints and just like the qualitative spatial reasoning approaches, it rapidly approaches intractability.

%\par{Developments like the \textit{Winthrop Method}~\cite{Keatley2021} for detecting secret caches leverage the idea that people remember locations based on the salient objects near them.
%The method includes a methodical search pattern from distinctive landmarks near a location to maximize the probability of finding the concealed cache. 
%Similarly, the geospatial search problem we address relies on recalling what objects were present, and how they are arranged to prune the search space.}
% 
%\par{A contemporary example of how much pruning power is derived from understanding the spatial arrangement of objects can be seen in the online game %\textit{Geoguessr}~\footnote{\href{https://www.geoguessr.com/}{https://www.geoguessr.com/}}.
%In Geoguessr, the player must determine their exact location in the world from a street view panorama, only using the clues available in the panorama. 
%The success of high-performing players of \textit{Geoguessr} in their last-mile search shows that by combining the relative positions of enough objects, even common ones like bus stops and crosswalks, have the power to substantially prune a large search space down to just a few, or even one candidate location~\cite{BernersLee2023}.
%The aspects of recalling locations accurately that Schwering's work identify are, unsurprisingly, driven by a strong ability for most people to recall the relative arrangement of objects to their position, and the order of encountering those objects when traveling~\cite{Schwering2014}.}

%
%The core spatial search functionality of \emph{GESTALT} leverages a pictorial query interface to bridge the gap between the user's mental model of objects, and the geospatial data available to search. 
%The idea of pictorial querying was developed by Soffer and Samet~\cite{Soffer1997}.
%Their approach focuses on matching locations from a user-defined pictorial input to an underlying database of maps. 
%In the 27 years since Soffer and Samet first specified the Pictorial Query Language, considerable advances in digital storage and access to geospatial data have addressed some of the initial scale limitations that the conversion of maps to digital pictorial representations presents. 
%For example, the role of a system like MAGELLAN~\cite{Samet1998} in \emph{GESTALT} is replaced by sourcing the locations from OSM, and the requirement for a separate system to efficiently index and query map tiles like MARCO~\cite{Samet1996} is now handled by OSM in its entirety. 
%Additional work in pictorial querying notes the problem of searching when there are multiple instances of the objects, which is NP-Complete when formulated as a subgraph matching problem~\cite{Folkers2000}. 
%\textit{GESTALT} addresses the complexity issues by successively pruning the search space, beginning with the last-mile search region, pruning out any locations that fail to contain the query terms, and then commencing the spatial search, which involves recursively pruning the search space even further, until a candidate location is determine to match or not match the configuration specified in the pictorial query.





%%%%%%%%%%%%%%%%%%%%%%%%%%%%%%%%%%%%%%%%%%%%%%%%%%%%%%%%%%%%
%\subsection{Geospatial Question Answering}
%Towards identifying where the user could be looking, the field of geospatial question answering, related to geospatial information retrieval, offers promising directions. 
%In 2018 Punjani et al. sought to determine whether geospatial information could be incorporated into a question-answering system~\cite{Punjani2018}. 
%They used the established Frankenstein variant of the Qanary approach to developing question-answering pipelines to develop GeoQA. 
%GeoQA can answer questions in several useful geospatial categories, including point, range, and property-based queries. 
%They built their system on linked data collected from OpenStreetMaps and WikiData. 
%More recently, the efforts to develop WorldKG, a geospatial knowledge graph of the world, extend the linked-data approach and allow users to generate SPARQL queries to answer complex geospatial questions across the fused knowledge of OpenStreetMaps, DBPedia and WikiData~\cite{dsouza2021}. 
%These linked data approaches to geospatial question answering are valuable advancements but do not include enough micro-terrain detail to satisfy the requirement of my project to allow a user to find a location based on partial information about the micro-terrain of the location.
%%%%%%%%%%%%%%%%%%%%%%%%%%%%%%%%%%%%%%%%%%%%%%%%%%%%%%%%%%%%

%Games like \textit{Geoguessr}~\footnote{\href{https://www.geoguessr.com/}{Geoguessr Website}} which invert the last-mile
%demonstrates that for many people, figuring out where they are in the world can be a source of much joy. 
%A 2023 journal article analyzing the strategies employed by a top player identifies several successful strategies that are used by a leading player~\cite{Berners-Lee2023}. 
%The Geoguessr problem is a distinct and, in many ways, reciprocal problem to the \textit{last-mile} search problem \textit{GESTALT} seeks to solve. While the last mile search assumes a general region is known, the geoguessr frequently has no idea where in the world they are and needs to use clues from the interface (powered by google street view) to determine the country, state, county etc., that they are in. 
%While not directly relevant, Geoguessr demonstrates the importance of searching for landmarks, and often even benign objects like bus stops, in locating where in the world an image is. 

%A 2013 Neuroscience study shows that when we revisit those locations, the objects we remember serve as keys to other memories of that location~\cite{Miller2013}, \textit{GESTALT} aims to exploit these geospatial aspects of memory for helping searchers to find the locations they are looking for.
%\textbf{Winthropping}. The Winthrop Method is a geospatial search method developed by Captain Winthrop of the Royal Engineers for use in Northern Ireland in the 1970s~\cite{Keatley2021} to detect clandestine weapons caches and concealed improvised explosive devices. 
%The underlying logic is that to find something, a human has to have some method of navigating to it and that the objects in our environment help to form mental models of the terrain, we can use to navigate by. 
%A popular example is the closing scene of the film \textit{The Shawshank Redemption} where the protagonist Andy Dufrense provides instructions to his fellow prisoner Ellis Redding on how to find a gift left for him \textit{"It's got a long rock wall with a big oak tree at the north end... find that spot. At the base of that wall, you'll find a ...piece of black, volcanic glass. There's some thing buried under it I want you to have."} 
%The two reasons that Winthropping works are \textit{Affordance} and \textit{Satisficing}. Affordance refers to the interaction of an agent with its environment and, in simple terms, means that particular objects will have a more significant impact on us and remain in our memory than others. 
%When combined with the idea of satisficing, in which an agent makes a satisfactory or sufficient but possibly sub-optimal decision, it illustrates the value of object-based search. 
%We cannot remember every detail of a location, so our brains will record only a few key objects or experiences for us to leverage. 
%A 2013 Neuroscience study shows that when we revisit those locations, the objects we remember serve as keys to other memories of that location~\cite{Miller2013}, \textit{GESTALT} aims to exploit these geospatial aspects of memory for helping searchers to find the locations they are looking for.


%\subsection{Remote Sensing Imagery}
%Relevant to the data collection subsystem if \textit{GESTALT}, 2018 efforts to improve the state of the art in object detection from remote sensing imagery focused on developing datasets for training and evaluating models. 
%The xView project supports the detection of 60 classes of objects~\cite{Lam2018} using horizontal bounding boxes. The DOTA project supports a much more modest 20 classes~\cite{Xia2018}. 
%Both focus towards shipping and industrial applications and so will not generalize well. The 2021 update to the DOTA project highlighted that remote sensing object detection continues to suffer from the arbitrary rotation of objects and the vast disparities in the clustering of objects~\cite{Ding2021}. 
%DOTA version 2 tries to address these issues by employing orientation bounding boxes but is still very constrained in the classes of objects it supports. 
%A separate effort by Li et al. in 2020 can detect objects less constrained to heavy industry but only accounts for 20 or so objects~\cite{Li2020}. 
%Overall current work on remote-sensing object detection indicates that it is still an emerging field incapable of supporting the labeling of micro-terrain features required for the proposed system. 
%An area growing parallel with remote sensing object detection is image captioning and visual question answering. 
%In addition to implementing previously discussed object detection techniques, they employ alternate data sources to augment their ability to provide answers to natural language questions about remote sensing imagery. 
%In 2017 Shi and Zou demonstrated that it is possible to automate caption generation for remote sensing imagery. However, their experimentation showed it to be ineffective at tasks like counting objects~\cite{Shi2017}, an essential requirement of micro-terrain analysis.

%\subsection{Visual Question Answering}
%The paper that initiated the domain of Remote Sensing Visual Question Answering (RSVQA) was published in 2020 by Lobry et al. and showed an excellent ability to answer direct questions about a given remote sensing image, including area estimates, object counts and determining the relative locations of objects~\cite{Lobry2020}. 
%These capabilities are all helpful in the concept mapping component of \textit{GESTALT}. Importantly they also incorporated geospatial information from OpenStreetMap into their system. 
%A fundamental limitation they identified is that the lack of information in OSM about specific micro-terrain features and the inability of object detection models to provide it presents a significant gap holding back the advancement of the RSVQA field. 
%My work may contribute towards closing that gap. Later work by Zheng et al. and Yuan et al. highlight that RSVQA is very much in its infancy, and they focus on improving the underlying models used in RSVQA systems~\cite{Zheng2021, Yuan2022}.
%In addition to the gap identified by Lobry et al., one limitation is that the RSVQA approach focuses on answering questions about the things the user is already looking at. 
%In my partial information use case, the user doesn't know exactly where to look, so using the RSVQA approach would render no improvement to performance over the visual inspection itself. 
%The real challenge is identifying where to look, not what they are already looking at.

%NSCH find other OSM-based terrain search tools and summarize them here


%In their use-case, they start with a photo and attempt to geolocate it by performing the \textit{Last-Mile} search based on the presence of OSM Map Features. Their system implements a more user-friendly interface over the OSM Overpass-Turbo API, allowing searchers to use drop-down lists and sliders rather than generating complicated queries. 
%Their tool demonstrates the importance and utility of the problem that \textit{GESTALT} seeks to solve. It also shows how much data is already contained on OSM that can be leveraged by \textit{GESTALT}. 
%The obvious limitation is in performance. As it directly queries the OSM API, the query executes very slowly. For example, running the query to return all wineries in the swan valley region took 21 seconds to complete. In \textit{GESTALT}, it takes less than 1 second. 
%Their system does not address data collection, ownership assignment or concept mapping. It also doesn't allow for spatial queries expected by \textit{GESTALT} searchers. 
%The Bellingcat tool demonstrates the utility of a system like \textit{GESTALT} and highlights how careful consideration of data structure design is essential for developing a performant system. 

%\subsection{Human Spatial Cognition}
 %"Global objects were already in their original locations and could not be moved. This was because (a) they had not been targets before, (b) it is rather unusual to move them in everyday life, and (c) a recent VR scene building experiment by Draschkow and Võ (2017) shows that they are usually placed earlier in the building process and then serve as anchors to guide the arrangement of smaller objects that we typically interact with, which makes them appropriate location recall cues in our study." \cite{Helbing2020}
 %robots that do perception need anchoring too \cite{Oliveira2016}

  %This study of cognitive strategies used in navigation and route planning shows that people form spatial representations of a virtual environment in differnt ways, including likely hierarchically, or with separate local and global features \cite{Weisberg2016}. The implications are that our approach is consistent with how people remember spatial locations using a hierarchical cognitive map of objects/landmarks and/or using a hierarchical representation of objects/landmarks. Explain how these two possibilities are both served by the GESTALT searching method...

      %representing and solving geospatial pattern matching queries are impractical for the task.


% \par{
%     In general, Spatial Keyword Queries (SKQ) accepts a collection of keywords and a user location and returns the object clusters that are spatially and textually relevant~\cite{Cao2012,Zhang2009} using algorithms like SKECa+~\cite{Guo2015} and the GESTALT system's \textit{basic}, \textit{ranked} and \textit{fuzzy} queries~\cite{Osul2023}.

%     }
    
    %While SKQ can return a location based on the presence of objects within a constrained distance that matches keyword queries, it does not allow for querying based on topological or directional relations.
%}
%\par{
%    Conventional pattern-based spatial querying approaches are dominated by the \textit{Metric} and \textit{Topological} relationships of spatial data, along with external information like object labels, or keywords. 

%    The keyword approaches avoid directional constraints because of how dense representing n-ary directional relationships is.

% \subsection{Spatial Joins}
% \par{
%     \textit{Spatial Joins} (SJ) are operations that given two collections of spatial objects find all pairs of objects satisfying some specified spatial constraints~\cite{Jacox2007}.
%     %SJs are commonly executed as a set-intersection, and the multi-constraint instantiation of spatial joining (MSJ) has been shown to successfully encode \textit{Metric, Topological} and \textit{Directional} constraints and solve for the join using a CSP~\cite{Papadias1998}.
%     %The efficiency of MSJ is highly dependent on the data structures being joined, and the method used to execute the `search' for suitable join conditions.
%     }


%Set-intersection problems suffer from low precision with small numbers of query terms, and so adding constraints in terms of \textit{distance, topology} and \textit{direction} is required to prune the result set.
}    


    %Papadias et. al. provide three algorithms derived from the Forward-Checking algorithm for Constraint Satisfaction Problems (CSP). One of the algorithms is very sensitive to the number of query objects, with the other two degrade as constraints become more numerous and dense. 
    %Their experiments show that their \textit{Multi-Forward-Checking} (MFC) algorithm runs exponentially to the number of query terms, while their \textit{Window-Reduction} and \textit{Joint-Window-Reduction} algorithms run close to linear to the number of query terms, but are still sensitive to the number of objects and constraints in the database being queried, and degrade rapidly with large numbers of constraints.
    %There has been significant work on improving the algorithmic performance of MSJ, and a recent focus has been on parallelizing and distributing computation to improve execution speed~\cite{Du2017}. 

    %MSJ can match based on topological, metric, and directional constraints but do not widely support keyword querying. 
    %MSJ are computationally intensive, and typically leverage another Spatial Pattern Matching approach as the 'solver' for the join. 
%}


    %The natural extension of these approaches is to formulate search as an isomorphic subgraph matching problem to strengthen topological matching~\cite{Folkers2000}, at the cost of $\mathcal{O}(m\times 2^m)$ complexity, where \textit{m} is the number of vertices in the database. 
    %\nrscomment{below paragraph simplifies to one sentence summary and see table for detailed complexity comparison}
%\par{    
    
    %
    %SGM is the natural approach to apply to the SPM problem based on the underlying graph representation common to SPM tasks.  
    %More modern approaches have continued to leverage SGM, for example, the SpaceKey system allows users to search for patterns of objects based on a combination of SKQ and metric (distance) relations and the inclusion or exclusion of certain objects within those distances which is encoded as a subgraph matching problem~\cite{Fang2018}.
    %Fang et. al. extend their work on using SGM for SPM by developing four algorithms~\cite{Fang2019}. 
    %Their \textit{Multi-Pair-Join (MPJ)} algorithm runs in $\mathcal{O}(m\zeta |D|^2+\xi)$ where $\zeta$ is a sampling threshold in range $[0,1]$ and $\xi$ is the maximal number of partial matches. 
    %Their \textit{Multi-Star-Join (MSJ)} algorithm runs in $\mathcal{O}(n^4+m|D|^2+\xi)$ time worst case, but practically its running time is faster than the MPJ. 
    %\textit{Top-K SPM} is used to refine the results set when recall is too high, and \textit{Partial SPM} is used when too few results are returned to increase the candidate pool by returning maximally partial matches. 
    %Both \textit{Top-K SPM} and \textit{PSPM} incur additional computational costs~\cite{Fang2019}.
    %The Efficient Spatial Pattern Matching Algorithm (ESPM) improves MPJ and MSJ by replacing the underlying IR-Tree with an Inverted Linear Quadtree. 
    %The Quadtree is more efficient when working on large search spaces but still uses graph matching as the underlying search technique. 
    %ESPM takes $\mathcal{O}(|D_w|^n)$ time, where $n$ is the number of vertices in the graph and $|D_w|$ is the set of objects returned by the SKQ component~\cite{Chen2019}. 

    %\osullikomment{is the K-Clique stuff even worth including? I'm not sure its relevant}
    
    %K-Clique detection in spatial data uses a nearest-neighbor graph, iterative pruning, and early stopping in a graph-matching approach. 
    %Its worst-case complexity is $\mathcal{O}(n^k)$ with expected complexity of $\mathcal{O}(NK)$~\cite{Taniguchi2022}.
%    }  

 %Graphs are the most natural structure to encode pairwise constraints that apply to an object set. 

 %\par{
    %The AI community has been interested in resolving constraint satisfaction problems for a long time.
    
    %}

     %in the worst case, but have expected performance much closer to linear in the number of query terms.
    %That observation is not universal, with applications like  proving exponential in the number of query terms~\cite{Papadias1998}.} 
    %In addition to Papadias' formulation of MSJ as a CSP~\cite{Papadias1998}, Dylla, et al. survey \textit{Qualitative Spatial Reasoning} (QSR) approaches finding that the 11 approaches that use directional relations as \textit{binary} constraints have at least $\mathcal{O}(n^3)$ complexity. 
    %They scope out \textit{n-ary} constraints as being generally intractable~\cite{Dylla2017}. 
    %
    %An exemplar QSR approach is the \textit{StarVars} system, which extends the earlier \textit{STAR} calculus~\cite{Renz2004} to support reasoning about the relative locations of objects to each other~\cite{Lee2013}.
    %They take the space-segmentation approach, and the space around each of the \textit{n} objects is divided into \textit{m} sectors, and an $\mathcal{O}(m^n)$ algorithm for constraint satisfaction is used, making the approach impractical for large query sets.}

%\par{
%        Some approaches combine CSP and SGM. For example, the \textit{Sketchmapia} system accepts a user sketch and extracts features of \textit{street segments}, \textit{junctions}, \textit{landmarks}, and \textit{city blocks} and takes the explicit link-encoding approach to formulate a \textit{Qualitative Constraint Network} that is then searched against a database of maps for matches using incremental sub-graph matching~\cite{Schwering2014,Jan2015}.
%}    

%\par{
%    We consider SPM as the underlying problem COMPASS sets out to solve. 
%    Broadly, we observe that directional relations offer powerful discrimination capabilities in spatial queries, but effectively representing and querying them is computationally prohibitive. 
    %
%    Traditional SKQ approaches are limited by their inability to represent directional relations. 
    %In the overarching GESTALT system, we implement SKQ as a filtering technique to prune the candidate locations being provided to our spatial pattern matching approaches~\cite{Osul2023}.
    %
%    Set-intersection approaches will only allow for a naive bucketing approach to spatial relationships, which we implement with our \textit{location-centric} query approach for comparison.
    %
%    SGM approaches are generally infeasible for matching directional relations because adequately representing all directional constraints requires at least a fully connected graph and graph-matching complexity scales to the density of the graph.
    %
%    CSP/QSR approaches experience similar difficulties to SGM, with their performance degrading rapidly as the number of constraints increases.
    %
%    Complexity increases further when there is no known global coordinate system to compare against. 
%    By viewing the problem from a user-centric perspective we need to not only match a pictorially-specified query pattern against the underlying database, but that query pattern is unlikely to be aligned with the global coordinate system underpinning the database. 
%    None of the surveyed problems are capable of efficiently supporting cardinality-invariant directional relation spatial pattern matching, and the COMPASS suite aims to offer a mechanism to fill this gap. 
%}
